% Created 2025-03-02 Sun 14:31
% Intended LaTeX compiler: pdflatex
\documentclass[11pt]{article}
\usepackage[utf8]{inputenc}
\usepackage[T1]{fontenc}
\usepackage{graphicx}
\usepackage{longtable}
\usepackage{wrapfig}
\usepackage{rotating}
\usepackage[normalem]{ulem}
\usepackage{amsmath}
\usepackage{amssymb}
\usepackage{capt-of}
\usepackage{hyperref}
\author{Anshul Sawant}
\date{\today}
\title{Homework 6}
\hypersetup{
 pdfauthor={Anshul Sawant},
 pdftitle={Homework 6},
 pdfkeywords={},
 pdfsubject={},
 pdfcreator={Emacs 29.4 (Org mode 9.6.15)}, 
 pdflang={English}}
\begin{document}

\maketitle
\tableofcontents


\section{Question 1.1}
\label{sec:org6cdec92}
\subsection{Prompt}
\label{sec:orgdc9bb8b}
Sekou writes the numbers 15, 16,17,18,19. After he erases one of his numbers, the sum of the remaining four numbers is a multiple of 4. Which number did he erase? [AMC8 2025]
\subsection{Model output}
\label{sec:org1cd5cdd}
First, calculate the sum of all the numbers Sekou wrote:

\[ 
15 + 16 + 17 + 18 + 19 = 85 
\]

Next, determine which number can be erased so that the sum of the remaining numbers is a multiple of 4. 

Let's subtract each number from the total sum and check:

\begin{itemize}
\item Erasing 15: \(85 - 15 = 70\) (which is divisible by 4)
\item Erasing 16: \(85 - 16 = 69\)
\item Erasing 17: \(85 - 17 = 68\) (which is divisible by 4)
\item Erasing 18: \(85 - 18 = 67\)
\item Erasing 19: \(85 - 19 = 66\)
\end{itemize}

The sums 70 and 68 are multiples of 4. Sekou can erase either 15 or 17 to achieve this.

\subsection{Result expected}
\label{sec:org894735a}
17

\subsection{Relative error rate}
\label{sec:orgef7f65b}
One of the answers (15) is off by 11.76\%

\subsection{Choice of model}
\label{sec:orgc235704}
GPT-4o

\section{Question 1.3 B: Eval vs Safe Eval}
\label{sec:org430e850}
Eval will allow arbitrary python code to be executed. It may be possible to get model to generate dangerous python code, for example, via in-context learning.
\section{Question 1.4: Accuracy with and without calculator}
\label{sec:org0ec497c}
Accuracy with calculator is 60.13\%. Accuracy without calculator is 35.22\%.

\section{Question 1.5 A: Incorrect answer without calculator}
\label{sec:org17892b7}
\subsection{Prompt}
\label{sec:org7e24f91}
Question: Since the townspeople requested that he makes a chocolate cake, he will need 306 lbs. of cocoa powder.
If the mayor already gave him 259 lbs., how much cocoa powder does he still need?
Answer:
\subsection{With calculator}
\label{sec:orgb46ab4c}
Question: Since the townspeople requested that he makes a chocolate cake, he will need 306 lbs. of cocoa powder.
If the mayor already gave him 259 lbs., how much cocoa powder does he still need?
Answer:\label{org9620837}47
\subsection{Without calculator}
\label{sec:org33d5440}
Question: Since the townspeople requested that he makes a chocolate cake, he will need 306 lbs. of cocoa powder.
If the mayor already gave him 259 lbs., how much cocoa powder does he still need?
Answer:\label{org74db53a}13
\subsection{Discussion}
\label{sec:org0a07f88}
While the model set up the computation correctly, it could not correctly compute the difference of the two numbers.

\section{Question 1.5 B: Incorrect answer with calculator}
\label{sec:orgca66291}
\subsection{Prompt}
\label{sec:org9d5eee1}
Question: Each side of a square kitchen tile is 7 inches long. What is the tile's area?
Answer:
\subsection{With calculator}
\label{sec:orgd90bee6}
Question: Each side of a square kitchen tile is 7 inches long. What is the tile's area?
Answer:\label{org5322700}28
\subsection{Without calculator}
\label{sec:org2a1d846}
Question: Each side of a square kitchen tile is 7 inches long. What is the tile's area?
Answer:\label{orgf4d1402}28
\subsection{Discussion}
\label{sec:orgfdd4b80}
The model probably did not understand the concept of area.
\end{document}
